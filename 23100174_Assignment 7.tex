
\documentclass{article}
\usepackage{enumerate}
\usepackage{graphicx,fancyhdr,amsmath,amssymb,amsthm,subfig,url,hyperref}
\usepackage[margin=1in]{geometry}
\newtheorem{theorem}{Theorem}

%----------------------- Student and Homework Information --------------------------

%%% PLEASE FILL THIS OUT WITH YOUR INFORMATION
\newcommand{\myname}{M.Shozab Hussain}
\newcommand{\myid}{23100174}
\newcommand{\hwNo}{Homework 7}
%%% END



\fancypagestyle{plain}{}
\pagestyle{fancy}
\fancyhf{}
\fancyhead[RO,LE]{\sffamily\bfseries\large LUMS}
\fancyhead[LO,RE]{\sffamily\bfseries\large CS-210 Discrete Mathematics}
\fancyfoot[LO,RE]{\sffamily\bfseries\large \myname: \myid @lums.edu.pk}
\fancyfoot[RO,LE]{\sffamily\bfseries\thepage}
\renewcommand{\headrulewidth}{1pt}
\renewcommand{\footrulewidth}{1pt}

%--------------------- This is the title of the document. DO NOT CHANGE IT ------------------------

\title{CS-210 \hwNo}
\author{\myname \qquad Student ID: \myid}

%--------------------------------- AFTER Entering the Student and Homework Information, write your answers below  ----------------------------------

\begin{document}
\maketitle
\begin{enumerate}[(1. ]
\Large

\item [1. ] 
Let P(n) be the statement that n dollars can be formed using just \$3 and \$7 bills \\
\underline{Base Case:} n= 12 = 3(4) + 7(0) \\
\underline{Base Case:} n=13 = 3(2) + 7(1) \\
\underline{Base Case:} n= 14 = 3(0) + 7(2) \\

\underline{IH:} we can express any value $i(12 \leq i \leq k)$ where $k \geq 14$ as $i=3a + 7b$ where a,b $ \in  \mathbb{Z}^+ $ \\

Assuming that inductive hypothesis is true, we can express \\
k+1 dollars as 3a + 7b  where a,b $ \in  \mathbb{Z}^+ $ \\ \\
To show P(k+1), we can can use P(k-2) because it is true by inductive hypothesis since $(12 \leq k-2 \leq k) $ \\ \\
\underline{Inductive Step:} \\
P(k-2) = 3a + 7b \\ 
P(k-2+3) = 3a + 7b +3 \\
P(k+1) = 3(a+1) + 7b \\
Therefore, by assuming $k-2$ and adding a \$3 bill, we can get to $k+1$  dollars. \\ \\
Both the basis step and the inductive step have been completed so by the principle of strong induction, the statement is true for every integer n greater than or equal to 12.
\pagebreak

\item [2. ] 
\begin{enumerate}
\item 
Let P(n) be the statement that $4^n -1$ is divisible by 3 \\ \\
\underline{Base Case:} n=1 $ \rightarrow 4^1 - 1$ = 3 \\
3 is divisible by 3 hence, base case is true. \\ 

\underline{IH:} Assume that P(n) is true. \\

\underline{Inductive Step:} \\ 
Using P(n), we prove that P(n+1) is true. \\
$ P(n+1) = 4^{n+1} - 1$ \\
$ P(n+1) = 4.4^{n} - 1$ \\
Now by IH $4^n$ is true while $4.4^n$ is a multiple of 4. \\
If we subtract 1 from $4.4^n$, it will become a multiple of 3. \\
Hence, $4.4^{n} - 1$ is a multiple of 3 \\ 
A multiple of 3 means that it is divisible by 3 \\
Therefore, P(n+1) is also divisible by 3 and thus, is true. \\
  
\item 
Let P(n) be the statement that $22^n -1$ is divisible by 3 \\ \\
\underline{Base Case:} n=1 $ \rightarrow 22^1 - 1$ = 21 \\
$21/3 = 7$ hence, divisible by 3. Base case is true. \\

\underline{IH:} Assume that P(n) is true. \\

\underline{Inductive Step:} \\
 Using P(n), we prove that P(n+1) is true. \\
$ P(n+1) = 22^{n+1} - 1$ \\
$ P(n+1) = 22.22^{n} - 1$ \\
Now by IH $22^n$ is true while $22.22^n$  is a multiple of 22 \\
If we subtract 1 from $22.22^n$, it will become a multiple of 21. \\
21 is itself a multiple of 3 so this means a multiple of 21 is also a multiple of 3. \\
Hence, $22.22^{n} - 1$ is a multiple of 3 \\ 
A multiple of 3 means that it is divisible by 3 \\
Therefore, P(n+1) is also divisible by 3 and thus, is true. \\
\end{enumerate}
\pagebreak
\item [3. ]
Let P(n) be the statement that n amounts of postage can be formed using just 3-cent and 10-cent stamps \\
\underline{Base Case:} n= 18 = 3(6) + 10(0) \\
\underline{Base Case:} n=19 = 3(3) + 10(1) \\
\underline{Base Case:} n= 20 = 3(0) + 10(2) \\

\underline{IH:} we can express any value $i(18 \leq i \leq k)$ where $k \geq 20$ as $i=3a + 10b$ where a,b $ \in  \mathbb{Z}^+ $ \\

Assuming that inductive hypothesis is true, we can express \\
k+1 as 3a + 10b  where a,b $ \in  \mathbb{Z}^+ $ \\ \\
To show P(k+1), we can can use P(k-2) because it is true by inductive hypothesis since $(18 \leq k-2 \leq k) $ \\ \\
\underline{Inductive Step:} \\
P(k-2) = 3a + 10b \\ 
P(k-2+3) = 3a + 10b +3 \\
P(k+1) = 3(a+1) + 10b \\
Therefore, by assuming $k-2$ and adding a 3-cent stamp, we can get to $k+1$  postage. \\ \\
Both the basis step and the inductive step have been completed so by the principle of strong induction, the statement is true for every integer n greater than or equal to 18.


\end{enumerate}
\end{document}
